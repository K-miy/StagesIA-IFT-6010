\documentclass[french]{article}
\usepackage[utf8]{inputenc}
\usepackage[T1]{fontenc}
\usepackage{babel}
\begin{document}
	Voici trois exemples de listes. On y retrouve une 
	liste sans numérotation, une liste avec numérotation 
	et finalement une liste avec description.
	
	\begin{itemize}
		\item en français, les premiers éléments d'une 
		liste se terminent par un point virgule;
		\item chaque élément commence par une minuscule;
		\item le dernier élément a un point.
	\end{itemize}
	
	\begin{enumerate}
		\item L’hydrogène est le 1\ier{} élément.
		\item L’hélium est le 2\ieme{} élément.
		\item Le lithium est le 3\ieme{} élément.
	\end{enumerate}
	
	\begin{description}
		\item[Mercure] a un flux de rayonnement solaire de 9126.6~W/m$^2$.
		\item[Vénus] a un flux de rayonnement solaire de 2613.9~W/m$^2$.
		\item[Terre] a un flux de rayonnement solaire de 1367.6~W/m$^2$.
	\end{description}
\end{document}	