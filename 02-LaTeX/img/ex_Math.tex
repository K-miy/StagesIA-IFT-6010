\documentclass[french]{article}
\usepackage[utf8]{inputenc}
\usepackage{babel}
\begin{document}
	
Le déploiement simple d'expressions mathématiques complexes est une des grandes forces de \LaTeX. Il est tout aussi aisé d'insérer une équation en mode texte (<< text style >>), donc qui s'intègre dans un paragraphe, telles que $V=RI$ ou $\mathbf{b} = \mathbf{A} \mathbf{x}$, que des équations en mode d'affichage hors texte (<< display style >>) sans numérotation,
\begin{displaymath}
   b_i = \sum_{j=1}^{2} a_{ij} x_{j},
\end{displaymath}

ou encore avec numérotation, telle que l'éq.~(\ref{eq:matrice}):
\begin{equation}
   \left[ \begin{array}{c} b_{1} \\ b_{2} \end{array} \right]
      = \left[ \begin{array}{cc} a_{11} & a_{12} \\ a_{21} & a_{22} \end{array} \right]
        \times
        \left[ \begin{array}{c} x_{1} \\ x_{2} \end{array} \right].
   \label{eq:matrice}
\end{equation}

L'équation affichée sans numérotation ne possède évidemment pas de numéro, on ne peut donc ni y définir une référence dynamique, ni y référer.
\end{document}
